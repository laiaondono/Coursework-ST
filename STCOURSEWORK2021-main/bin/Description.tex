\documentclass{article}
\usepackage[utf8]{inputenc}
\PassOptionsToPackage{hyphens}{url}\usepackage{hyperref}
\usepackage[dvipsnames]{xcolor}
\usepackage{listings}

\title{ST COURSEWORK DESCRIPTION}
%\author{s.akca }
%\date{February 2021}

\usepackage{natbib}
\usepackage{graphicx}

\begin{document}

\maketitle

\section{SOFTWARE TESTING: PRACTICAL}
This page describes the practical for the Informatics Software Testing
course. It will be marked out of 100 points, and is worth 45\% of the
assessment of the course.

\section{DEADLINE}
The coursework comprises 3 tasks with the following issue dates and
deadlines:

\begin{itemize}
\item Issued: 26th February
\item \textbf{Deadline : 26th March, 4pm GMT}.
  % you had ``April'' but this is the date in Learn
\end{itemize}

The policy for late submission has been modified this year because of
the pandemic, rather than following the UG3 course guide
(\url{http://www.inf.ed.ac.uk/teaching/years/ug3/CourseGuide/coursework.html})

\textbf{Coursework will be scrutinised for plagiarism and academic
  misconduct.} Information on academic misconduct and advice for good
scholarly conduct is available at
\url{https://web.inf.ed.ac.uk/infweb/admin/policies/academic-misconduct}.


\section{TOOLS}

You can choose to use the Eclipse IDE, another IDE such as IntelliJ
IDEA, or just to use JUnit and other appropriate tools of your choice
standalone. I have no strong preference -- many people find the tools
available in Eclipse useful (if you haven't used Eclipse before maybe
now is the time to give it a try). You can discuss your choice and
problems you come across with the tools (without sharing specifics of
your coursework solutions) on Piazza, but note that tool-wrangling is
part of what this coursework is intended to help you learn so it is
ultimately \textbf{your responsibility} to get your chosen tools to
work. Don't leave it to the last minute!

The rest of this handout is written mostly as though you were using
Eclipse.

You will need some of the following:

\begin{enumerate}
\item JUnit 4. If necessary you can download JUnit from
  \url{https://junit.org/junit4/}. If you are using Eclipse it is
  probably already installed in the IDE. This article:
  \url{https://www.vogella.com/tutorials/JUnit4/article.html} is a
  reasonable introduction to using JUnit4 with Eclipse.
% The previous URL actually went to a JUnit5 tutorial! Since this url
% leads to the page that covers JUnit4 including with Eclipse, I think
% it will be fine on its own.
  
    \item You will need some kind of coverage analysis tool:
    \begin{itemize}
        \item In Eclipse you can use EclEmma:
          \url{https://www.eclemma.org/}. If you use Eclipse on DICE,
          or if you install the version of Eclipse that is current at
          the time of writing (2020-12) and select ``for Java
          developers'' when asked, this should already be installed,
          and is accessible as ``coverage'' from the toolbar or Run
          menu.  If not, it's easy to install through Eclipse's built
          in software update mechanism. IntelliJ IDEA also has a
          built-in coverage module.
        \item For stand-alone coverage you should consider something
          like Cobertura: \url{http://cobertura.github.io/cobertura/}
        \item A review of other OpenSource code coverage tools for
          Java is available at
          \url{https://java-source.net/open-source/code-coverage}.
    \end{itemize}
\end{enumerate}

Most of the tasks have an associated tutorial which will help you
prepare for it. Please prepare in advance for the tutorial to get the
most out of it.

\section{SETTING UP}

\subsection{Preparation}

If you don't have Eclipse installed and want to use it, you should
download it and install it. You can find Eclipse at
\url{https://eclipse.org/users/}. You want ``for Java developers''
when asked. This will come with JUnit and EclEmma
pre-installed. Eclipse's help menu leads to a lot of information, some
of which is outdated: you may wish also to consult the links given
above.

%% Perdita took this out, because that link leads to the overall
%% documentation. The top Getting Started tutorial doesn't mention
%% JUnit; the one under Java development user guide does, but is very
%% unhelpfully in terms of JUnit 3 and JDK 1.4!

%% Once you have installed Eclipse, you
%% should look at the tutorial
%% (\url{https://help.eclipse.org/2020-12/index.jsp}). Do enough of the
%% ``getting started'' tutorial that you have JUnit as a project in
%% Eclipse.

%% You should also install eclemma
%% (\url{https://www.eclemma.org/installation.html#updatesite}) if you
%% don't have it and intend to use it. You can delay this since it is not
%% essential for the first task.

%% You should spend some time looking at the JUnit project in Eclipse and
%% become familiar with its structure.

% Please use JUnit 4 as your testing environment.

\section{TASKS}

\subsection{TASK 1: FUNCTIONAL TESTING (25 POINTS)}
In this task you will implement JUnit tests using the specification
provided in the Github repository
(\url{https://github.com/SoftwareTestingEdinburgh/STCOURSEWORK2021}). The
repository also provides the implementation as a JAR file,
\texttt{ST\textunderscore COURSEWORK.jar}, so you can execute your
JUnit tests and observe test results. The specification is described
in detail, with helpful examples where necessary, in the
\texttt{Specifications.pdf} file.

Functional testing is a black box testing technique, so use the
specification file to derive tests and \textbf{NOT} the source
code. The JAR file can be used to execute the tests derived from the
specification. We have also provided a sample JUnit test case,
\texttt{CommandLineParserTest.java} file, to illustrate a typical test
case for the implementation in \texttt{ST\textunderscore
  COURSEWORK.jar}. All the files referred to above can be found at the
Github repository
(\url{https://github.com/SoftwareTestingEdinburgh/STCOURSEWORK2021}).

For this task, our \texttt{ST\textunderscore COURSEWORK.jar} file
contains 13 different bugs.
%% Perdita replaced ``vulnerabilities'' by ``bugs'', because the
%% former seems an odd term to use for something not security related
%% - don't think it's standard in the course.
The estimated difficulty of finding these bugs ranges from easy (5
bugs), to medium (4 bugs), to hard (4 bugs). Easy bugs will rewarded
as 1 point, Medium bugs will be rewarded as 2 points, and Hard bugs
will be rewarded as 3 points. If you are able to find all of them, you
will get full points(25) from this task.
%% Perdita added:
``Finding'' a bug means writing a test which should pass according to
the specification, but fails on the provided implementation.

\paragraph{Deliverables:}
\begin{itemize}
    \item A file Task1\textunderscore Functional.java
      that contains your JUnit tests.
\end{itemize}
 
\paragraph{SUBMISSION:}

Submit on Learn, just as you have for tutorials.

\section{TASK 2: COVERAGE ANALYSIS (30 POINTS)}

\subsection{TASK2.1: Analyzing Code Coverage (10 POINTS)}

The goal of this task is to measure and analyze the branch coverage of
the STCOURSEWORK2021 code achieved by executing the
test cases developed in Task 1.

%% Perdita: this seems weird because they've just created and
%% submitted that file, so where are they copying it from? Is my
%% alternative wording above OK?
%% The easiest way to achieve this with
%% EclEmma is by copying the test cases into
%% Task1\textunderscore Functional.java file which is
%% already in the project directory.

\paragraph{Deliverables}

\begin{itemize}
\item A file \texttt{task2-1.jpg}
%% Perdita changed the filename, because in Windows you cannot have a
%% file name that contains two dots.
  containing a screenshot of the branch coverage report as shown by
  the coverage measurement tool. Please make sure that the total
  branch coverage is clearly visible in the screenshot.
\end{itemize}

\paragraph{SUBMISSION:}

On Learn.

\subsection{TASK2.2: Generate Automated Test Cases with EvoSuite (15 Points) }

For this task, you will use EvoSuite to generate test cases for the
project. EvoSuite is a tool that automatically generates test cases
with assertions for classes written in Java code. To achieve this,
EvoSuite applies a novel hybrid approach that generates and optimizes
whole test suites towards satisfying a coverage criterion. You can
find detailed information at \url{https://www.evosuite.org/}.

For a modern Java version, you will need to use the latest version,
1.1.0, of evosuite. At the time of writing, this is not available as a
maven plugin, so you will need to follow the instructions for using
evosuite from the commandline. Then, if you are using a coverage tool
in an IDE, you will need to import the generated tests into the IDE.

%% Perdita: I don't think this is right. For a modern Java version
%% they need to use evosuite 1.1.0 and this is not available anywhere
%% as a maven plugin. So I think they need to use 1.1.0 from the
%% command line - right?
%% Our Java
%% project type is \textbf{Maven project}. So, basically you need to
%% follow the link
%% \url{https://www.evosuite.org/documentation/maven-plugin/} for further
%% changes in \textbf{pom.xml} file to measure the branch code coverage
%% of the tests which are generated by EvoSuite. You may need to do a
%% tiny change in EvoSuite test files. For this tiny change, you can
%% follow
%% \url{https://www.evosuite.org/documentation/measuring-code-coverage/}.

\paragraph{Deliverables:}

\begin{itemize}
\item A folder \texttt{Task2-2}. This folder should contain
\begin{enumerate}
\item A file \texttt{task2-2.jpg} which shows the branch coverage
  report achieved by EvoSuite test cases. Please make sure that the
  total branch coverage is clearly visible in the screenshot.
\item All of the generated test case files that were generated by
  EvoSuite.
\end{enumerate}
\end{itemize}

\paragraph{SUBMISSION:}

On Learn.

\subsection{TASK2.3: Advantages and Disadvantages of automated test
  generation Tool (5 points) }

In this task, you will state three advantages and disadvantages of
using automated test generation tool. Focus on questions such as:

\begin{itemize}
\item Why do we use automated test generation in the real world?
\item How effective is it to use test generation tools for big
  projects?
\end{itemize}

\paragraph{Deliverables:}
A \emph{plain text} file \texttt{Task2\textunderscore 3.txt} with a
brief reflection, organized as a list of bullet points, of advantages
and disadvantages of using automated test generation tools such as
EvoSuite.

\paragraph{SUBMISSION:}

On Learn.

\section{Task 3: Adding Functionality with Test-Driven Development (45
    points)} 

A TDD approach is typically interpreted as ``A programmer taking a TDD
approach refuses to write a new function until there is first a
\textbf{test that fails} because that function is not present.'' TDD
makes the programmer think through requirements or design before they
write functional code. Once the test is in place the programmer
proceeds to complete the implementation and checks if the test suite
now passes. Please keep in mind that your new code may break several
existing tests as well as the new one. The code and tests may need to
be refactored till the test suite passes and the specification is
fully implemented. In this task, you will need to follow the TDD
approach to implement and support a new additional specification in
the existing implementation. The new additional specification is
described in the last two pages of the specification file, available
in Github repository
(\url{https://github.com/SoftwareTestingEdinburgh/STCOURSEWORK2021}).


\paragraph{Deliverables:}
This task will involve a 2 part submission.

\begin{enumerate}
\item \textbf{Part 1}: Tests

  Submit \textbf{only your new JUnit tests} in a file named
  \texttt{Task3\textunderscore TDD\textunderscore 1.java}, for the new
  additional specification (last 2 pages of the specification
  document). Please check to make sure all of these new tests fail on
  the existing implementation available in the \texttt{src} folder.
    
\item \textbf{Part 2}: Implementation + Tests

  This part requires that you add source code to \texttt{Parser.java}
  in the \texttt{src} folder to support the new
  specification. Check whether all the tests you developed in Part 1
  pass for your modified implementation. If they don't, modify the
  implementation and/or tests so the entire test suite passes and the
  new specification is implemented correctly. Name your modified
  collection of new tests \texttt{Task3\textunderscore
    TDD\textunderscore2.java} (even if it is, in fact, identical to
  the file you submitted as \texttt{Task3\textunderscore
    TDD\textunderscore 1.java} in Part 1). Submit both the modified
  implementation of \texttt{Parser.java} and the test suite
  \texttt{Task3\textunderscore TDD\textunderscore2.java}.

 \end{enumerate}
 
 \textbf{PS: For Task 3, your require implementation need only modify
   the file \texttt{Parser.java}. Please do not touch other java
   files.}
 
 \paragraph{SUBMISSION:}

On Learn.
 
\end{document}
